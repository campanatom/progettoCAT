\documentclass[a4paper, 11pt]{article}
\usepackage[margin=3cm]{geometry}
\usepackage[]{fontenc}
\usepackage[utf8]{inputenc}
\usepackage[italian]{babel}
\usepackage{geometry}
\geometry{a4paper, top=2cm, bottom=3cm, left=1.5cm, right=1.5cm, heightrounded, bindingoffset=5mm}
\usepackage{amsmath}
\usepackage{amssymb}
\usepackage{gensymb}
\usepackage{graphicx}
\usepackage{psfrag,amsmath,amsfonts,verbatim}
\usepackage{xcolor}
\usepackage{color,soul}
\usepackage{fancyhdr}
\usepackage{indentfirst}
\usepackage{graphicx}
\usepackage{newlfont}
\usepackage{amssymb}
\usepackage{amsmath}
\usepackage{latexsym}
\usepackage{amsthm}
%\usepackage{subfigure}
\usepackage{subcaption}
\usepackage{psfrag}
\usepackage{footnote}
\usepackage{graphics}
\usepackage{color}
\usepackage{hyperref}
\usepackage{tikz}


\usetikzlibrary{snakes}
\usetikzlibrary{positioning}
\usetikzlibrary{shapes,arrows}

	
	\tikzstyle{block} = [draw, fill=white, rectangle, 
	minimum height=3em, minimum width=6em]
	\tikzstyle{sum} = [draw, fill=white, circle, node distance=1cm]
	\tikzstyle{input} = [coordinate]
	\tikzstyle{output} = [coordinate]
	\tikzstyle{pinstyle} = [pin edge={to-,thin,black}]

\newcommand{\courseacronym}{CAT}
\newcommand{\coursename}{Report\\Controlli Automatici - T}
\newcommand{\tipology}{A}
\newcommand{\trace}{1}
\newcommand{\projectname}{Controllo trattamento farmacologico su popolazioni cellulari}
\newcommand{\group}{37}

%opening
\title{ \vspace{-1in}
		\huge \strut \coursename \strut 
		\\
		\Large  \strut Progetto Tipologia \tipology - Traccia \trace 
		\\
		\Large  \strut \projectname\strut
		\\
		\Large  \strut Gruppo \group\strut
		\vspace{-0.4cm}
}
\author{Tommaso Campana, Giorgio Di Nicola, Alessio Orsini}
\date{\dots}

\begin{document}

\maketitle
\vspace{-0.5cm}

Il progetto riguarda il controllo di cellule in un ambiente farmacologico, la dinamica delle quali viene descritta dalle seguenti equazioni differenziali 

%
\begin{subequations}\label{eq:system}
\begin{align}
\dot{n}_s &= r_s \left( 1 - \frac{n_s + n_r}{K} \right) n_s - m_s c_f n_s - \beta n_s + \gamma n_r - \alpha c_f n_s, \tag{1a} \\
\dot{n}_r &= r_r \left( 1 - \frac{n_s + n_r}{K} \right) n_r - m_r c_f n_r + \beta n_s - \gamma n_r + \alpha c_f n_s. \tag{1b} 
\end{align}
\end{subequations}
%

dove \( n_s(t) \) rappresenta una densità di cellule suscettibili al trattamento
farmacologico, mentre \( n_r(t) \) indica, al contrario, una densità di cellule \( n_r(t) \) resistenti. 
I parametri \( r_s, r_r \in \mathbb{R} \) rappresentano i tassi di riproduzione delle due tipologie, e il parametro \( K \in \mathbb{R} \) rappresenta la densità massima di cellule che l’ambiente può contenere. La variabile d’ingresso \( c_f(t) \) indica la concentrazione del farmaco.
I parametri \( m_s, m_r \in \mathbb{R} \) determinano, rispettivamente, la mortalità delle cellule suscettibili e quella delle cellule resistenti, con \( m_s > m_r \). 
Tipicamente, le cellule possono mutare da una tipologia all’altra. Ad esempio, le cellule suscettibili possono diventare resistenti, come tenuto in conto dai termini \( -\beta n_s \) nella prima equazione e \( \beta n_s \) nella seconda equazione, con \( \beta \in \mathbb{R} \).
Analogamente accade per le cellule resistenti attraverso il termine \( \gamma n_r \), con \( \gamma \in \mathbb{R} \). Infine, il termine \( \alpha c_f n_r \) tiene conto delle cellule suscettibili che mutano in resistenti a seguito del trattamento farmacologico.


\section{Espressione del sistema in forma di stato e calcolo del sistema linearizzato intorno ad una coppia di equilibrio}

Innanzitutto, esprimiamo il sistema~\eqref{eq:system} nella seguente forma di stato
%
\begin{subequations}
\begin{align}\label{eq:state_form}
	\dot{x} &= f(x,u)
	\\
	y &= h(x,u).
\end{align}
\end{subequations}
%
Pertanto, andiamo individuare lo stato $x$, l'ingresso $u$ e l'uscita $y$ del sistema come segue 
%
\begin{align*}
	x := 
    \begin{bmatrix} x_1 \\ x_2\end{bmatrix},\quad con \quad x_1=n_s(t), \:\: x_2=n_r(t)
\end{align*}
\begin{align*}
    u := c_f(t), \quad y := x_2(t).
\end{align*}
%
Coerentemente con questa scelta, ricaviamo dal sistema~\eqref{eq:system} la seguente espressione per le funzioni $f$ ed $h$:
%
\begin{gather*}
	f(x,u) := \begin{bmatrix} f_1(x_1,x_2,u) \\ f_2(x_1,x_2,u)\end{bmatrix} = 
    \begin{bmatrix} 
    r_s \left( 1 - \frac{x_1 + x_2}{K} \right) x_1 - m_s x_1 u- \beta x_1 + \gamma x_2 - \alpha u x_1
    \\
    r_r \left( 1 - \frac{x_1 + x_2}{K} \right) x_2 - m_r x_2 u + \beta x_1 - \gamma x_2 + \alpha u x_1
    \end{bmatrix}
	\\
    \\
	h(x) := x_2.
\end{gather*}
%
Una volta calcolate $f$ ed $h$ esprimiamo~\eqref{eq:system} nella seguente forma di stato
%
\begin{subequations}\label{eq:our_system_state_form}
\begin{gather}
	\begin{bmatrix}
		\dot{x}_1
		\\
        \dot {x}_2
	\end{bmatrix} = 
    \begin{bmatrix}
	r_s \left( 1 - \frac{x_1 + x_2}{K} \right) x_1 - m_s x_1 u- \beta x_1 + \gamma x_2 - \alpha u x_1
    \\
    r_r \left( 1 - \frac{x_1 + x_2}{K} \right) x_2 - m_r x_2 u + \beta x_1 - \gamma x_2 + \alpha u x_1
	\end{bmatrix}  
    \label{eq:state_form_1}
	\\
	y = x_2.
\end{gather}
\end{subequations}
%
Per trovare la coppia di equilibrio $(x_e, u_e)$ di~\eqref{eq:our_system_state_form}, andiamo a risolvere il seguente sistema di equazioni:
%
\begin{align}
	\dot{x}_1=r_s \left( 1 - \frac{x_{e1} + x_{e2}}{K} \right) x_{e1} - m_s x_{e1} u- \beta x_{e1} + \gamma x_{e2} - \alpha u x_{e1} = 0 
    \\
    \dot{x}_2=r_r \left( 1 - \frac{x_{e1} + x_{e2}}{K} \right) x_{e2} - m_r x_{e2} u + \beta x_{e1} - \gamma x_{e2} + \alpha u x_{e1} = 0.
\end{align}
%
Nella tabella fornita dal testo sono già presenti i valori di equilibrio di \(n_{s,e}\) e \(n_{r,e}\), corrispondenti rispettivamente a \(x_{e1}\) e \(x_{e2}\).
Sostituendo i valori conosciuti e risolvendo il sistema di equazioni, otteniamo:
%
\begin{align}
	x_e := 
    \begin{bmatrix} x_{e1} \\ x_{e2} \end{bmatrix} = 
    \begin{bmatrix} 100 \\ 100 \end{bmatrix},
    \quad u_e = \dots.\label{eq:equilibirum_pair}
\end{align}
%
Definiamo le variabili alle variazioni $\delta x$, $\delta u$ e $\delta y$ come 
%
\begin{align*}
	\delta x &= \dots, 
	\quad
	\delta u = \dots, 
	\quad
	\delta y = \dots.
\end{align*}
%
L'evoluzione del sistema espressa nelle variabili alle variazioni pu\`o essere approssimativamente descritta mediante il seguente sistema lineare
%
\begin{subequations}\label{eq:linearized_system}
\begin{align}
	\delta \dot{x} &= A\delta x + B\delta u
	\\
	\delta y &= C\delta x + D\delta u,
\end{align}
\end{subequations}
%
dove le matrici $A$, $B$, $C$ e $D$ vengono calcolate come
%
\begin{subequations}\label{eq:matrices}
\begin{align}
	A &= \dots
	\\
	B &= \dots
	\\
	C &= \dots
	\\
	D &= \dots.
\end{align}
\end{subequations}
%
\section{Calcolo Funzione di Trasferimento}

In questa sezione, andiamo a calcolare la funzione di trasferimento $G(s)$ dall'ingresso $\delta u$ all'uscita $\delta y$ mediante la seguente formula 
%
%
\begin{align}\label{eq:transfer_function}
G(s) = \dots = \dots.
\end{align}
%
Dunque il sistema linearizzato~\eqref{eq:linearized_system} è caratterizzato dalla funzione di trasferimento~\eqref{eq:transfer_function} con $\dots$ poli $p_1 = \cdots, \cdots$ e $\dots$ zeri $z_i =\cdots$. In Figura \dots mostriamo il corrispondente diagramma di Bode.

\dots

\begin{figure}[h]
    \centering
    \fbox{\parbox{0.35\textwidth}{\vspace{1.9cm} Placeholder \vspace{1.9cm}}}
    \caption{Caption.}
    \label{fig:bode_diagram}
\end{figure}

\dots\\

Inoltre, \dots

\dots

\dots

\dots


\section{Mappatura specifiche del regolatore}
\label{sec:specifications}

Le specifiche da soddisfare sono
\begin{itemize}
	\item[1)] \dots\\
	\item[2)] \dots\\
	....\\
	\item[6)] \dots.
\end{itemize}
%
Andiamo ad effettuare la mappatura punto per punto le specifiche richieste. \dots  

Pertanto, in Figura \dots, mostriamo il diagramma di Bode della funzione di trasferimento $G(s)$ con le zone proibite emerse dalla mappatura delle specifiche.

\dots

\begin{figure}[h]
    \centering
    \fbox{\parbox{0.35\textwidth}{\vspace{1.9cm} Placeholder \vspace{1.9cm}}}
    \caption{Caption.}
    \label{fig:bode_diagram}
\end{figure}

\dots

Si può notare che \dots

\dots

\section{Sintesi del regolatore statico}
\label{sec:static_regulator}

In questa sezione progettiamo il regolatore statico $R_s(s)$ partendo dalle analisi fatte in sezione~\ref{sec:specifications}.

\dots

Dunque, definiamo la funzione estesa $G_e(s) = R_s(s)G(s)$ e, in Figura \dots, mostriamo il suo diagramma di Bode per verificare se e quali zone proibite vengono attraversate.

\dots

\begin{figure}[h]
    \centering
    \fbox{\parbox{0.35\textwidth}{\vspace{1.9cm} Placeholder \vspace{1.9cm}}}
    \caption{Caption.}
    \label{fig:bode_diagram}
\end{figure}

\dots
 
Da Figura \dots, emerge \dots\\

\dots\\

Inoltre, possiamo notare che \dots

\dots

\dots

\dots


\section{Sintesi del regolatore dinamico}

In questa sezione, progettiamo il regolatore dinamico $R_d(s)$. 
%
Dalle analisi fatte in Sezione~\ref{sec:static_regulator}, notiamo di essere nello Scenario di tipo \dots. Dunque, progettiamo $R_d(s)$ riccorrendo a \dots


In Figura \dots, mostriamo il diagramma di Bode della funzione d'anello $L(s) = R_d(s) G_e(s)$

\begin{figure}[h]
    \centering
    \fbox{\parbox{0.35\textwidth}{\vspace{1.9cm} Placeholder \vspace{1.9cm}}}
    \caption{Caption.}
    \label{fig:bode_diagram_L}
\end{figure}

\dots

Possiamo notare che \dots

\dots

\dots

\dots

\section{Test sul sistema linearizzato}

In questa sezione, testiamo l'efficacia del controllore progettato sul sistema linearizzato con \dots

In Figura \dots, mostriamo lo schema a blocchi del sistema in anello chiuso. \dots

\begin{figure}[h]
    \centering
    \fbox{\parbox{0.35\textwidth}{\vspace{1.9cm} Placeholder \vspace{1.9cm}}}
    \caption{Caption.}
    \label{fig:bode_diagram_L}
\end{figure}

\dots

Di seguito è riportato \dots in merito alla risposta del sistema a fronte di un ingresso \dots

\begin{figure}[h]
    \centering
    \fbox{\parbox{0.35\textwidth}{\vspace{1.9cm} Placeholder \vspace{1.9cm}}}
    \caption{Caption.}
    \label{fig:bode_diagram_L}
\end{figure}

\dots
Si nota che \dots

\dots

Inoltre possiamo notare dalle seguenti figure \dots che i disturbi \dots 

\begin{figure}[h]
    \centering
    \fbox{\parbox{0.35\textwidth}{\vspace{1.9cm} Placeholder \vspace{1.9cm}}}
    \caption{Caption.}
    \label{fig:bode_diagram_L}
\end{figure}

In seguito, \dots

\dots

\dots

\dots

\section{Test sul sistema non lineare}

In questa sezione, testiamo l'efficacia del controllore progettato sul modello non lineare con \dots \\

In Figura \dots, mostriamo lo schema a blocchi del sistema in anello chiuso. \dots

\begin{figure}[h]
    \centering
    \fbox{\parbox{0.35\textwidth}{\vspace{1.9cm} Placeholder \vspace{1.9cm}}}
    \caption{Caption.}
    \label{fig:bode_diagram_L}
\end{figure}

\dots

Di seguito è riportato \dots in merito alla risposta del sistema a fronte di un ingresso \dots

\begin{figure}[h]
    \centering
    \fbox{\parbox{0.35\textwidth}{\vspace{1.9cm} Placeholder \vspace{1.9cm}}}
    \caption{Caption.}
    \label{fig:bode_diagram_L}
\end{figure}

\dots
Si nota che \dots\\

\dots

Rispetto alle simulazioni riguardanti il sistema linearizzato emerge \dots\\

\dots

Inoltre, è possibile osservare \dots\\

\dots

\dots

\dots


\section{Punti opzionali}

\subsection{Primo punto}

\dots 

\subsection{Secondo punto}

\dots

\subsection{Terzo punto}

\dots

\section{Conclusioni}

\dots

\end{document}
